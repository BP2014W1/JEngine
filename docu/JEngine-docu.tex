% THIS IS SIGPROC-SP.TEX - VERSION 3.1
% WORKS WITH V3.2SP OF ACM_PROC_ARTICLE-SP.CLS
% APRIL 2009
%
% For tracking purposes - this is V3.1SP - APRIL 2009

\documentclass{acm_proc_article-sp}

\begin{document}

%\title{Docu zur JEngine}
\title{Documentation ..}
\subtitle{[a written report]
\titlenote{This elaboration was produced as part of the ``PCM'' at the chair of Business Process Technology supervised by Prof. Dr. Matthias Weske}
}
\numberofauthors{3} 

%
%
\author{
\alignauthor
NJP\\
       \affaddr{Hasso Plattner Institute}\\
       \affaddr{Prof.-Dr.-Helmert-Str. 2-3}\\
       \affaddr{14482 Potsdam, Germany}\\
       \email{\footnotesize  NJP@student.hpi.de}
}
\date{28 December 2014}

\maketitle

%
%
\begin{abstract}
This documentation contains a reflection of the original paper ``PCM'' from 2013 including an overview of the topic, metrics and a reality check.
\end{abstract}

\category{Information Systems Applications}{Big Data}{Real Time Analysis}
\category{Software Engineering}{Psychoinformatics}[fraud security, complexity measures, performance measures]
\terms{Theory}
\keywords{Recommender Systems, Predictable Behavioural Analysis, User Profiling} 

%
%
\section{Introduction}


%\begin{figure}
%\centering
%%\psfig{file=rosette.ps, height=1in, width=1in,}
%\includegraphics[width=3in]{../img/2-Use-Cases/news-recommendations.png}
%\caption{One possible example of Collaborative Filtering by calculating news recommendations.}
%\label{fig:example_news_recommendation}
%\end{figure}





%
%
\bibliographystyle{abbrv}
%\bibliographystyle{unsrt}
\bibliography{JEngine-docu} 

%
%
\end{document}
