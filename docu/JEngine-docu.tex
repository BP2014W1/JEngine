% THIS IS SIGPROC-SP.TEX - VERSION 3.1
% WORKS WITH V3.2SP OF ACM_PROC_ARTICLE-SP.CLS
% APRIL 2009
%
% For tracking purposes - this is V3.1SP - APRIL 2009

\documentclass{acm_proc_article-sp}

\begin{document}

%\title{Docu zur JEngine}
\title{Documentation ..}
\subtitle{[a written report]
\titlenote{This elaboration was produced as part of the ``PCM'' at the chair of Business Process Technology supervised by Prof. Dr. Matthias Weske}
}
\numberofauthors{3} 

%
%
\author{
\alignauthor
NJP\\
       \affaddr{Hasso Plattner Institute}\\
       \affaddr{Prof.-Dr.-Helmert-Str. 2-3}\\
       \affaddr{14482 Potsdam, Germany}\\
       \email{\footnotesize  NJP@student.hpi.de}
}
\date{28 December 2014}

\maketitle

%
%
\begin{abstract}
This documentation contains a reflection of the original paper ``PCM'' from 2013 including an overview of the topic, metrics and a reality check.
\end{abstract}

\category{Information Systems Applications}{Big Data}{Real Time Analysis}
\category{Software Engineering}{Psychoinformatics}[fraud security, complexity measures, performance measures]
\terms{Theory}
\keywords{Recommender Systems, Predictable Behavioural Analysis, User Profiling} 

%
%
\section{Introduction}

%
%
\section{MetaModell}

%\begin{figure}
%\centering
%%\psfig{file=rosette.ps, height=1in, width=1in,}
%\includegraphics[width=3in]{../img/2-Use-Cases/news-recommendations.png}
%\caption{One possible example of Collaborative Filtering by calculating news recommendations.}
%\label{fig:example_news_recommendation}
%\end{figure}

%
%
\section{JEngine}
Overall JEngine

%
%
\subsection{JCore}

%
%
\subsection{JComparser}

%
%
\subsection{JFrontEnd}

%
%
\subsection{JDatabase}

%
%
\section{Processeditor}

\begin{markdown}
# PCM modelling using the Processeditor

This document explains how to use the Processeditor to create PCM models.
A PCM-Process can be described by many PCM fragments and one PCM scenario.

## Preparations

Currently you need both, the Processeditor Workbench and the Processeditor Server to model and Save PCM.
You will use the Workbench for modelling and the Server as a global repository.

## PCM Fragments

PCM Fragments are small Business Process models. They can be modelled using a subset of the BPMN-Notation:

* Tasks
* Events
** Blanko Start-Event
** Blanko End-Event
* Gateways
** Parallel Gateway
** Exclusive Gateway
* Data Objects
* Sequence Flow
* Data Flow

All this elements are offered by the model type PCM Fragment.

### Marking a Task as Global

PCM allows to use the same task in more than one fragment. To do so

1. model the Task (in one scenario)
2. Save the model to the repository
3. Right click on the Task and choose *Properties*
4. Set the *global flag*

### Copy and Refer an existing Task

1. In another Fragment right click on any node
2. Choose "Copy and Refer Task"
3. Connect to the server if necessary
4. Choose the Model and the Task you want to refer
5. Click on Ok

## PCM Scenario

A Scenario defines which PCM Fragments are part of one Process.
All PCM Fragments have to be saved on the Server. You can alter the Scenario only by moving the nodes and adding/removing PCM Fragments.

### Defining a PCM Scenario

1. Create a new PCM Scenario Model.
2. Right Click on one of the two nodes
3. Choose Add Fragments
4. Mark all Models you want to add in the left List (CTRL for multi select)
5. click on add than on ok

Now there should be entries for all the fragments (inside green node) and for all their data objects (inside white node).

### Removing a Fragment From an Scenario

1. Right Click on one of the two nodes
2. Choose *Add Fragments*
3. Select all the models you want to remove from the right list
4. Click on *Remove* than click *Ok*

### Set a Termination Condition

If a termination condition is full filled the process is terminated.
Currently only one termination condition consisting of one Data Object in one specific state is possible.

1. Open your Scenario
2. Right Click on the canvas (not the Nodes)
3. Choose *Properties*
4. Fill out the *Termination Data Object* and *Termination State* fields

### Copy and Alter a Complete Fragment

You can create a variation of an existing PCM Fragment using the Plug-in *Create Variant*.

1. First click on *Plug-Ins*
2. Choose *Create Variant*
3. Choose your Fragment and click on *Ok*
\end{markdown}


%
%
\subsection{Processeditor Server}


%
%
\subsection{Processeditor Client}


%
%
\bibliographystyle{abbrv}
%\bibliographystyle{unsrt}
\bibliography{JEngine-docu} 

%
%
\end{document}
