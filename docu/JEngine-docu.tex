% THIS IS SIGPROC-SP.TEX - VERSION 3.1
% WORKS WITH V3.2SP OF ACM_PROC_ARTICLE-SP.CLS
% APRIL 2009
%
% For tracking purposes - this is V3.1SP - APRIL 2009

\documentclass{acm_proc_article-sp}

\begin{document}

\title{Documentation zur JEngine}
\subtitle{[a written report]
\titlenote{This elaboration was produced as part of the ``PCM'' at the chair of Business Process Technology supervised by Prof. Dr. Matthias Weske}
}
\numberofauthors{1} 

%
%
\author{
\alignauthor
NJP\\
       \affaddr{Hasso Plattner Institute}\\
       \affaddr{Prof.-Dr.-Helmert-Str. 2-3}\\
       \affaddr{14482 Potsdam, Germany}\\
       \email{\footnotesize  NJP@student.hpi.de}
}
\date{28 December 2014}

\maketitle

%
%
\begin{abstract}
This documentation contains a reflection of the original paper ``PCM'' from 2013 including an overview of the topic, metrics and a reality check.
\end{abstract}

\category{Information Systems Applications}{Big Data}{Real Time Analysis}
\category{Software Engineering}{Psychoinformatics}[fraud security, complexity measures, performance measures]
\terms{Theory}
\keywords{Recommender Systems, Predictable Behavioural Analysis, User Profiling} 

%
%
\section{Introduction}

%
%
\section{MetaModell}

%\begin{figure}
%\centering
%%\psfig{file=rosette.ps, height=1in, width=1in,}
%\includegraphics[width=3in]{../img/2-Use-Cases/news-recommendations.png}
%\caption{One possible example of Collaborative Filtering by calculating news recommendations.}
%\label{fig:example_news_recommendation}
%\end{figure}

%
%
\section{JEngine}
Overall JEngine

%
%
\subsection{JCore}

%
%
\subsection{JComparser}

%
%
\subsection{JFrontEnd}

%
%
\subsection{JDatabase}

%
%
\section{Processeditor}

\section{PCM modelling using the
Processeditor}\label{pcm-modelling-using-the-processeditor}

This document explains how to use the Processeditor to create PCM
models. A PCM-Process can be described by many PCM fragments and one PCM
scenario.

\subsection{Preparations}\label{preparations}

Currently you need both, the Processeditor Workbench and the
Processeditor Server to model and Save PCM. You will use the Workbench
for modelling and the Server as a global repository.

\subsection{PCM Fragments}\label{pcm-fragments}

PCM Fragments are small Business Process models. They can be modelled
using a subset of the BPMN-Notation:

\begin{itemize}
\itemsep1pt\parskip0pt\parsep0pt
\item
  Tasks
\item
  Events ** Blanko Start-Event ** Blanko End-Event
\item
  Gateways ** Parallel Gateway ** Exclusive Gateway
\item
  Data Objects
\item
  Sequence Flow
\item
  Data Flow
\end{itemize}

All this elements are offered by the model type PCM Fragment.

\subsubsection{Marking a Task as Global}\label{marking-a-task-as-global}

PCM allows to use the same task in more than one fragment. To do so

\begin{enumerate}
\def\labelenumi{\arabic{enumi}.}
\itemsep1pt\parskip0pt\parsep0pt
\item
  model the Task (in one scenario)
\item
  Save the model to the repository
\item
  Right click on the Task and choose \emph{Properties}
\item
  Set the \emph{global flag}
\end{enumerate}

\subsubsection{Copy and Refer an existing
Task}\label{copy-and-refer-an-existing-task}

\begin{enumerate}
\def\labelenumi{\arabic{enumi}.}
\itemsep1pt\parskip0pt\parsep0pt
\item
  In another Fragment right click on any node
\item
  Choose ``Copy and Refer Task''
\item
  Connect to the server if necessary
\item
  Choose the Model and the Task you want to refer
\item
  Click on Ok
\end{enumerate}

\subsection{PCM Scenario}\label{pcm-scenario}

A Scenario defines which PCM Fragments are part of one Process. All PCM
Fragments have to be saved on the Server. You can alter the Scenario
only by moving the nodes and adding/removing PCM Fragments.

\subsubsection{Defining a PCM Scenario}\label{defining-a-pcm-scenario}

\begin{enumerate}
\def\labelenumi{\arabic{enumi}.}
\itemsep1pt\parskip0pt\parsep0pt
\item
  Create a new PCM Scenario Model.
\item
  Right Click on one of the two nodes
\item
  Choose Add Fragments
\item
  Mark all Models you want to add in the left List (CTRL for multi
  select)
\item
  click on add than on ok
\end{enumerate}

Now there should be entries for all the fragments (inside green node)
and for all their data objects (inside white node).

\subsubsection{Removing a Fragment From an
Scenario}\label{removing-a-fragment-from-an-scenario}

\begin{enumerate}
\def\labelenumi{\arabic{enumi}.}
\itemsep1pt\parskip0pt\parsep0pt
\item
  Right Click on one of the two nodes
\item
  Choose \emph{Add Fragments}
\item
  Select all the models you want to remove from the right list
\item
  Click on \emph{Remove} than click \emph{Ok}
\end{enumerate}

\subsubsection{Set a Termination
Condition}\label{set-a-termination-condition}

If a termination condition is full filled the process is terminated.
Currently only one termination condition consisting of one Data Object
in one specific state is possible.

\begin{enumerate}
\def\labelenumi{\arabic{enumi}.}
\itemsep1pt\parskip0pt\parsep0pt
\item
  Open your Scenario
\item
  Right Click on the canvas (not the Nodes)
\item
  Choose \emph{Properties}
\item
  Fill out the \emph{Termination Data Object} and \emph{Termination
  State} fields
\end{enumerate}

\subsubsection{Copy and Alter a Complete
Fragment}\label{copy-and-alter-a-complete-fragment}

You can create a variation of an existing PCM Fragment using the Plug-in
\emph{Create Variant}.

\begin{enumerate}
\def\labelenumi{\arabic{enumi}.}
\itemsep1pt\parskip0pt\parsep0pt
\item
  First click on \emph{Plug-Ins}
\item
  Choose \emph{Create Variant}
\item
  Choose your Fragment and click on \emph{Ok}
\end{enumerate}


%
%
\subsection{Processeditor Server}


%
%
\subsection{Processeditor Client}


%
%
\bibliographystyle{abbrv}
%\bibliographystyle{unsrt}
\bibliography{JEngine-docu} 

%
%
\end{document}
