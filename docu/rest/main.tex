\documentclass[paper=a4]{scrartcl}
\usepackage{hyperref}
\usepackage{listings}
\usepackage{morefloats}
\usepackage[section] {placeins}
\usepackage{caption}
\usepackage{todonotes} %todo notes

%%%%%%%%%%%%%%%%%%%%%%%%% Configuration %%%%%%%%%%%%%%%%%%%%%%
\lstset{
    basicstyle=\footnotesize,
    tabsize=2
}
\DeclareCaptionType{json}[Example][List of JSON Examples]
%\newenvironment{json}{}{}
%%%%%%%%%%%%%%%%%%%%%%%%%% Header %%%%%%%%%%%%%%%%%%%%%%%%%% 

%opening
\title{JEngine REST API Specification v2.0}
\author{BP2014W1 - Team}

\begin{document}

\maketitle

%\listoftodos %displaying open todos
\setcounter{totalnumber}{5}
%%%%%%%%%%%%%%%%%%%%%%%%%% Abstract %%%%%%%%%%%%%%%%%%%%%%%%%% 

%
%%
\begin{abstract}
	One goal of the JEngine was to provide a flexible API.
	Developers can use this API to build an UI or to control/use the engine from within another application.
	There are methods to configure, control and monitor processes and their instances.
	This document will provide a specification/documentation of the Rest-Interface, by providing the information about the allowed methods and examples .
\end{abstract}

%%%%%%%%%%%%%%%%%%%%%%%%%% Overall API Structure %%%%%%%%%%%%%%%%%%%%%%%%%% 

%
%%
\section{Structure}
    \begin{centering}
		\label{tbl:structure}
		\centering
		\begin{tabular}{r|r}
			\texttt{http://domain:port/jengine/api/}&\texttt{interface/\{version\}}\\
			&\texttt{config/\{version}\}\\
			&\texttt{jcomparser/\{version\}}\\
			&\texttt{history/\{version\}}\\
			&\texttt{analytics/\{version\}}
		\end{tabular}
		\captionof{table}{API Core Path}
	\end{centering}
	The API is divided into four parts (Table \ref{tbl:structure}):
	\begin{enumerate}
		\item \textbf{Interface.}
			  The interface provides methods for creating and controlling instances of scenarios.
			  More information can be found in Chapter \ref{subsec:Interface}.
		\item \textbf{Config.}
			  This API can be used to configure a process model.
			  Currently the configuration is restricted to email tasks.
			  More information and examples can be found in Chapter \ref{subsec:Config}.
		\item \textbf{Comparser.}
			  The JEngine can be used to execute Production Case Management (PCM) models created with the \href{BP2014W1 processeditor.html}{JEDItor}\footnote{\url{BP2014W1 processeditor.html}}.
			  The JComparser is an interface between the core of the engine and the model repository provided by the JEDItor.
			  The JComparser provides and interface to view and load models from the repository.
			  More information and examples can be found in chapter \ref{subsec:Comparser}.
		\item \textbf{History.}
			  Detailed information about state changes of data object and activities of running and terminated scenario instances can be accessed via the history interface.
			  Chapter \ref{subsec:History} provides examples and detailed information.
	    \item \textbf{Analytics.}
			  to be explained. (framework for implementing analytic algorithms for analysing the PCM execution logs)
	\end{enumerate}

%%%%%%%%%%%%%%%%%%%%%%%%%% Functions %%%%%%%%%%%%%%%%%%%%%%%%%% 

%
%%
\section{Functions}

	This chapter provides information about all possible REST calls.
	For more detailed information you may take a look at the \textit{Java doc} provides for each class.

%%%%%%%%%%%%%%%%%%%%%%%%%% Interface %%%%%%%%%%%%%%%%%%%%%%%%%% 
\subsection{Interface}
\label{subsec:Interface}

	The interface provides methods mainly used by the UI.
	It offers functions for viewing and controlling scenarios and their instances.
	
	%
    %%
	\subsubsection{Scenarios}
	
	%%%----------
	\textbf{Get all Scenarios}\\
			\begin{tabular}{lll}
				http & GET & \texttt{interface/\{version\}/scenario}
			\end{tabular}
		\begin{flushleft}
			\begin{lstlisting}
{
	"ids":[140,141,142,143],
	"links":{
		"140":"http://domain:port/jengine/api/interface/v2/scenario/140",
		"141":"http://domain:port/jengine/api/interface/v2/scenario/141",
		"142":"http://domain:port/jengine/api/interface/v2/scenario/142",
		"143":"http://domain:port/jengine/api/interface/v2/scenario/143"
	},
	"labels":{
		"140":"TestScenario",
		"141":"CoffeeScenario",
		"142":"BookingScenario",
		"143":"BoschScenario"
	}
}
			\end{lstlisting}
			\captionof{json}{Example output of get all Scenarios request}
		\end{flushleft}
	%%%----------
	\textbf{Get all Scenarios with filter condition}\\
    		\begin{tabular}{lll}
				http & GET & \texttt{interface/\{version\}/scenario?filter=Coffee}
			\end{tabular}
		\begin{flushleft}
			\begin{lstlisting}
{
	"ids":[141],
	"links":{
		"141":"http://domain:port/jengine/api/interface/v2/scenario/141"
	},
	"labels":{
		"141":"CoffeeScenario"
	}
}
			\end{lstlisting}
			\captionof{json}{Example output of get all Scenarios request with filter condition}
		\end{flushleft}
	%%%----------
	\textbf{Get detailed information about a Scenario}\\
			\begin{tabular}{lll}
				http & GET & \texttt{interface/\{version\}/scenario/1}
			\end{tabular}
		\begin{flushleft}
			\begin{lstlisting}
{
	"modelid":0,
	"instances":"http://domain:port/jengine/api/interface/v2/scenario/1/instance",
	"name":"HELLOWORLD",
	"id":1,
	"modelversion":0
}
			\end{lstlisting}
			\captionof{json}{Example output of get a specific scenario}
		\end{flushleft}
	%
	%%
	\subsubsection{Scenario Instances}
	%%%----------
	\textbf{Get all scenario instances}
	    Scenario instances, that are in state terminated will not be listed.\\
			\begin{tabular}{lll}
				http & GET & \texttt{interface/\{version\}/scenario/1/instance}
    		\end{tabular}
		\begin{flushleft}
			\begin{lstlisting}
{
	"ids":[47,48,49],
	"links":{
		"47":"http://domain:port/jengine/api/interface/v2/scenario/1/instance/47",
		"48":"http://domain:port/jengine/api/interface/v2/scenario/1/instance/48",
		"49":"http://domain:port/jengine/api/interface/v2/scenario/1/instance/49"
	},
	"labels":{
		"47":"Berlin-Istanbul",
		"48":"Herr Meyer Reise",
		"49":"Sommerurlaub"
	}
}
			\end{lstlisting}
			\captionof{json}{Example output of get all scenario instances}
		\end{flushleft}
	%%%----------
	\textbf{Get all scenario instances with filter condition}\\
			\begin{tabular}{lll}
				http & GET & \texttt{interface/\{version\}/scenario/1/instance?filter=Herr}
			\end{tabular}
		\begin{flushleft}
			\begin{lstlisting}
{
	"ids":[47,48,49],
	"links":{
		"48":"http://domain:port/jengine/api/interface/v2/scenario/1/instance/48"
	},
	"labels":{
		"48":"Herr Meyer Reise"
	}
}
			\end{lstlisting}
			\captionof{json}{Example output of get all scenario instances with condition}
		\end{flushleft}
	%%%----------
	\textbf{Start a new instance}\\
	    A new instance can be started by performing a PUT or POST. The PUT provides the possibility to name the new instance.\\
			\begin{tabular}{lll}
				http & PUT & \texttt{interface/\{version\}/scenario/1/instance}\\
				http & POST & \texttt{interface/\{version\}/scenario/1/instance}
			\end{tabular}\\
	    \textbf{If you want to name the scenario instance you have to provide a body (Figure \ref{fig:body}):}\\
		\begin{flushleft}
			\begin{lstlisting}
{
	"name":"This is the new instance"
}
			\end{lstlisting}
			\captionof{json}{Body of the PUT request to create a new named instance}
		    \label{fig:body}
		\end{flushleft}
	%%%----------
    \textbf{Get detailed information about a scenario instance}\\
			\begin{tabular}{lll}
				http & GET & \texttt{interface/\{version\}/scenario/1/instance/48}
			\end{tabular}
		\begin{flushleft}
			\begin{lstlisting}
{
    "name":"Herr Meyer Reise",
    "id":48,
    "terminated":false,
    "scenario_id":1,
    "activities":"http://domain:port/jengine/api/interface/v2/scenario/1/instance/
                                                                        48/activity"
}
			\end{lstlisting}
			\captionof{json}{Example output of get a single scenario instance}
		\end{flushleft}
	\subsubsection{Activity Instances}
	\textbf{List all activity instances}\\
			\begin{tabular}{lll}
				http & GET & \texttt{interface/\{version\}/scenario/1/instance/48/activity}
			\end{tabular}
		\begin{flushleft}
			\begin{lstlisting}
{
    "activities":{
        "189":{
            "link":"http://domain:port/jengine/interface/v2/scenario/1/instance/
                                                                48/activity/189",
            "id":189,
            "label":"Book journey",
            "state":"ready"},
        "6686":{
            "link":"http://domain:port/jengine/interface/v2/scenario/1/instance/
                                                               48/activity/6686",
            "id":6686,
            "label":"Cancel journey",
            "state":"ready"
        },
        "ids":[189,6686]
}
    		\end{lstlisting}
    		\captionof{json}{Example of get all activities}
        \end{flushleft}
    \textbf{Get Activities with State}\\
			\begin{tabular}{lll}
				http & GET & \texttt{interface/\{version\}/scenario/1/instance/48/
				                                              activity?state=ready}
			\end{tabular}
		\begin{flushleft}
			\begin{lstlisting}
{
    "activities":{
        "189":{
            "link":"http://domain:port/jengine/interface/v2/scenario/1/instance/
                                                                 48/activity/189",
            "id":189,
            "label":"Book journey",
            "state":"ready"},
        "6686":{
            "link":"http://domain:port/jengine/interface/v2/scenario/1/instance/
                                                                48/activity/6686",
            "id":6686,
            "label":"Cancel journey",
            "state":"ready"
        },
        "ids":[189,6686]
}
    		\end{lstlisting}
    		\captionof{json}{Example of get all activities with an specific state}
        \end{flushleft}
    \textbf{Get Activities with Filter}\\
			\begin{tabular}{lll}
				http & GET & \texttt{interface/\{version\}/scenario/1/instance/48/activity?filter=Cancel}
			\end{tabular}
		\begin{flushleft}
			\begin{lstlisting}
{
    "activities":{
        "6686":{
            "link":"http://domain:port/jengine/interface/v2/scenario/1/instance/
                                                                48/activity/6686",
            "id":6686,
            "label":"Cancel journey",
            "state":"ready"
        },
        "ids":[6686]
}
    		\end{lstlisting}
    		\captionof{json}{Example of get all activities with an specific filter condition}
        \end{flushleft}
        \begin{centering}
        \centering
            \begin{tabular}{|l|l|}
                \hline
                Parameter & Options\\
                \hline
                state & ready, running, termianted  \\
                \hline
                filter & a String \\
                \hline
            \end{tabular}
        \captionof{table}{Query Parameters and their options}
        \label{tab:options_states}
        \end{centering}
    \begin{flushleft}
        % why is this stil centered?
        \textbf{Change the state of an activity instance}\\
    \end{flushleft}
			\begin{tabular}{lll}
				http & PUT & \texttt{interface/\{version\}/scenario/1/instance/48/activity?state=begin}
			\end{tabular}
        \begin{centering}
        \centering
            \begin{tabular}{|l|l|}
                \hline
                Parameter & Options\\
                \hline
                state & begin, terminate  \\
                \hline
            \end{tabular}
        \captionof{table}{Query Parameters and their options}
        \label{tab:options_statechanges}
        \end{centering}
		\begin{flushleft}
			\begin{lstlisting}
{
    "message":"activity state changed."
}
		\end{lstlisting}
    		\captionof{json}{Example output of begin activity}
       \end{flushleft}
       
    \subsubsection{Data objects}
    
    This path provides methods to get information about Data Object instance used inside an scenario instance.\\
    \textbf{Get all Data Objects of an instance}\\
        	\begin{tabular}{lll}
        		http & GET & \texttt{interface/\{version\}/scenario/1/instance/48/
        		                                              dataobject}
        	\end{tabular}
		\begin{flushleft}
			\begin{lstlisting}
{
  "ids":[1,2],
  "results":{
      "1":{
        "link":"http://server:port/interface/v2/scenario/1/instance/72/dataobject/1",
        "id":1,
        "label":"Flight",
        "state":"payed"
      },
      "2":{
          "link":"http://server:port/interface/v2/scenario/1/instance/72/dataobject/2",
          "id":2,
          "label":"Reservation",
          "state":"init"
      }
  }
}
		\end{lstlisting}
    		\captionof{json}{Example output of get all Dataobjects}
       \end{flushleft}
    \textbf{Get detailed information about a Data Objects instance}\\
        	\begin{tabular}{lll}
        		http & GET & \texttt{interface/\{version\}/scenario/1/instance/48/
        		                                              dataobject/1}
        	\end{tabular}
		\begin{flushleft}
			\begin{lstlisting}
{
    "id":1,
    "label":"object1",
    "state":"Flight"
}
		\end{lstlisting}
    		\captionof{json}{Example output of get all Dataobjects}
       \end{flushleft}
       
    \textbf{Terminations Bedingungen}\\
			\begin{tabular}{lll}
				http & PUT & \texttt{interface/\{version\}/scenario/1/instance/48/
				                                              activity?state=begin}
			\end{tabular}
		\begin{flushleft}
			\begin{lstlisting}
{
  "conditions":{
  "1":[{"data_object":"Flight","set_id":1,"state":"payed"},
      {"data_object":"Reservation","set_id":1,"state":"payed"}]
  "2":[{"data_object":"Flight","set_id":2,"state":"canceled"},
      {"data_object":"Reservation","set_id":2,"state":"canceled"}]
  },
  "setIDs":[1,2]
}
		\end{lstlisting}
    		\captionof{json}{Example output of get all Dataobjects}
       \end{flushleft}
       
           \textbf{Get Output}\\
			\begin{tabular}{lll}
				http & GET & \texttt{interface/\{version\}/scenario/1/instance/2/activity/8587/output}
			\end{tabular}
		\begin{flushleft}
			\begin{lstlisting}
[
    {
        "id":209,
        "linkDataObject":"http://172.16.64.113:8080/JEngine/api/interface/v2/scenario/160/instance
        /1406/outputset/209"
    }
]
		\end{lstlisting}
    		\captionof{json}{Example output of get output for activities}
       \end{flushleft}
       
        \textbf{Get References}\\
			\begin{tabular}{lll}
				http & GET & \texttt{interface/\{version\}/scenario/1/instance/2/activity/8587/references}
			\end{tabular}
		\begin{flushleft}
			\begin{lstlisting}
{
    "ids":[],
    "activities":[]
}
		\end{lstlisting}
    		\captionof{json}{Example output of get references for activities}
       \end{flushleft}
       
       
        \textbf{Get Outputset}\\
			\begin{tabular}{lll}
				http & GET & \texttt{interface/\{version\}/scenario/1/instance/2/activity/outputset/209}
			\end{tabular}
		\begin{flushleft}
			\begin{lstlisting}
[
    {
        "label":"Bestellung",
        "id":1184,
        "state":"init",
        "attributeConfiguration":
            [
                {
                    "id":298,
                    "name":"Preis",
                    "type":"",
                    "value":""
                },
                {
                    "id":299,
                    "name":"Artikel",
                    "type":"",
                    "value":""
                }
            ]
    }
]
		\end{lstlisting}
    		\captionof{json}{Example output of get outputset}
       \end{flushleft}
\newpage
		
%%%%%%%%%%%%%%%%%%%%%%%%%% JConfig %%%%%%%%%%%%%%%%%%%%%%%%%% 
%
%%
\subsection{JConfiguration alias Config}  %status = done   GET(4/4) PUT(2/2) DELETE (1/1)
\label{subsec:Config}

	%
    %%
	\subsubsection{Service eMail Tasks}
	

	\textbf{Get all available email tasks}\\
			\begin{tabular}{lll}
				http & GET & \texttt{config/\{version\}/scenario/148/emailtask/}
			\end{tabular}\\
		\begin{flushleft}
			\begin{lstlisting}
{
    "ids":
        [
            448
        ]
}
			\end{lstlisting}
			\captionof{json}{Example output of get all available email tasks}
		\end{flushleft}
	
	
	 	\textbf{Get details for a specific email task}\\
			\begin{tabular}{lll}
				http & GET & \texttt{config/\{version\}/scenario/1/emailtask/448/?}
			\end{tabular}\\
		\begin{flushleft}
			\begin{lstlisting}
{
    "receiver":"bp2014w1@byom.de",
    "subject":"Test",
    "message":"Test Message"
}
			\end{lstlisting}
			\captionof{json}{Example output of get all available email tasks}
		\end{flushleft}


	\textbf{PUT changes for an email task}\\
			\begin{tabular}{lll}
				http & PUT & \texttt{config/\{version\}/emailtask/448/?}
			\end{tabular}\\
		\begin{flushleft}
			\begin{lstlisting}
{
    "receiver":"bp2014w1@byom.de",
    "subject":"Test",
    "message":"Test Message"
}
			\end{lstlisting}
			\captionof{json}{Example output of PUT request to change an email task}
		\end{flushleft}

	 
	%
    %%
	\subsubsection{Scenarios}
	
	
	\textbf{Delete an scenario}\\
			\begin{tabular}{lll}
				http & DELETE & \texttt{config/\{version\}/scenario/156/?}
			\end{tabular}\\


	%
    %%
	\subsubsection{Webservice Tasks}
	
		\textbf{GET all webservice tasks}\\
			\begin{tabular}{lll}
				http & GET & \texttt{config/\{version\}/scenario/1/webservice?}
			\end{tabular}\\
		\begin{flushleft}
			\begin{lstlisting}
{
    "ids":[527]
}
			\end{lstlisting}
			\captionof{json}{Example output of get all webservice tasks for a specific scenario}
		\end{flushleft}

	
		\textbf{GET specific webservice task}\\
			\begin{tabular}{lll}
				http & GET & \texttt{config/\{version\}/scenario/1/webservice/253?}
			\end{tabular}\\
		\begin{flushleft}
			\begin{lstlisting}
{
    "body":"",
    "link":"http://localhost:9998/interface/v2/scenario/156/instance",
    "method":"POST",
    "attributes":
        [
            {
                "order":1,
                "controlnode_id":527,
                "key":"id",
                "dataattribute_id":14
            }
        ]
}
			\end{lstlisting}
			\captionof{json}{Example output of get specific webservice tasks for a specific scenario}
		\end{flushleft}
		
		
		\textbf{PUT specific webservice task}\\
			\begin{tabular}{lll}
				http & PUT & \texttt{config/\{version\}/webservice/253?}
			\end{tabular}\\
		\begin{flushleft}
			\begin{lstlisting}
{
"method":"POST",
"link":"http://localhost:9998/interface/v2/scenario/156/instance",
"attributes":
    [
        {
            "order":1,
            "controlnode_id":527,
            "key":"id",
            "dataattribute_id":14
        },
        {
            "order":2,
            "controlnode_id":527,
            "key":"link",
            "dataattribute_id":15
        }
    ]
}
			\end{lstlisting}
			\captionof{json}{Example output of get specific webservice tasks for a specific scenario}
		\end{flushleft}
\newpage

%%%%%%%%%%%%%%%%%%%%%%%%%% JComparser %%%%%%%%%%%%%%%%%%%%%%%%%% 
%
%%
\subsection{JComparser}  %status = done   GET(2/2) POST(1/1) 
\label{subsec:Comparser}

	%
    %%
	\subsubsection{Scenarios}
	
	\textbf{Get all available scenarios for import}\\
			\begin{tabular}{lll}
				http & GET & \texttt{jcomparser/scenarios}
			\end{tabular}
		\begin{flushleft}
			\begin{lstlisting}
{
    "ids":
        {
            "800682779972007735":"Coffee process",
            "635432698":"Coffee process",
            "932221086":"Kaffeeprozess"
        }
}
			\end{lstlisting}
			\captionof{json}{Example output of get all available scenarios from jcomparser}
		\end{flushleft}

	 	\textbf{Get image of scenario}\\
			\begin{tabular}{lll}
				http & GET & \texttt{jcomparser/scenarios/932221086/image/}
			\end{tabular}
		\begin{flushleft}
			\begin{lstlisting}
                \missingfigure{example scenario image}
			\end{lstlisting}
			\captionof{json}{Example output of get activies log}
		\end{flushleft}


	 	\textbf{POST launch import of scenario}\\
			\begin{tabular}{lll}
				http & POST & \texttt{jcomparser/scenarios/932221086/}
			\end{tabular}

\newpage
	 
%%%%%%%%%%%%%%%%%%%%%%%%%% jHistory %%%%%%%%%%%%%%%%%%%%%%%%%% 
%
%%
\subsection{JHistory alias History} %status = done   GET(3/3)
\label{subsec:History}

	%
    %%
	\subsubsection{Activities}
	
	\textbf{Get activities log}\\
			\begin{tabular}{lll}
				http & GET & \texttt{history/\{version\}/scenario/149/instance/1085/activities/}
			\end{tabular}
		\begin{flushleft}
			\begin{lstlisting}
{
    "881":
        {
            "activityinstance_id":7871,
            "timestamp":"2015-03-17 16:26:29.0",
            "scenarioinstance_id":1085,
            "label":"Kaffeestand pruefen",
            "newstate":"init"
        },
    [..]
}
			\end{lstlisting}
			\captionof{json}{Example output of get activies log}
		\end{flushleft}

	%
    %%
	\subsubsection{Data Objects}
	
	 	\textbf{Get dataobjects log}\\
			\begin{tabular}{lll}
				http & GET & \texttt{history/\{version\}/scenario/149/instance/1085/dataobjects/}
			\end{tabular}\\
		\begin{flushleft}
			\begin{lstlisting}
{
    "68":
        {
            "timestamp":"2015-03-17 16:26:56.0",
            "scenarioinstance_id":1085,
            "dataobjectinstance_id":879,
            "new_state_name":"fertig",
            "name":"Kaffee",
            "old_state_id":120,
            "old_state_name":"gekocht",
            "new_state_id":119
        },
    [..]
}
			\end{lstlisting}
			\captionof{json}{Example output of get dataobjects log}
		\end{flushleft}
		%	\caption{Example output of get dataobjects log}
		%\end{figure}\\
	 %%%----------
	 
	 	%
    %%
	\subsubsection{Attributes}
	
	\textbf{Get attribute log}\\
			\begin{tabular}{lll}
				http & GET & \texttt{history/\{version\}/scenario/149/instance/1085/attributes/}
			\end{tabular}
		\begin{flushleft}
			\begin{lstlisting}
{
    "1":
        {
            "h.scenarioinstance_id":1,
            "da.name":"Preis",
            "h.timestamp":"2015-05-07 09:43:33.0",
            "h.id":1,
            "h.dataattributeinstance_id":1,
            "h.newvalue":"",
            "do.name":"Bestellung"
        },
}			\end{lstlisting}
			\captionof{json}{Example output of get activies log}
		\end{flushleft}
\newpage

%%%%%%%%%%%%%%%%%%%%%%%%%% jAnaltics %%%%%%%%%%%%%%%%%%%%%%%%%% 
%
%%
\subsection{JAnalytics} %status = open   GET(0/1) POST(0/1)
\label{subsec:JAnalytics}

	%
    %%
	\subsubsection{Analytic Algorithms}
	
	%\textbf{Get all services}\\
	The names of all registered algorithms executable can be obtained via\\
			\begin{tabular}{lll}
				http & GET & \texttt{analytics/\{version\}/services}
			\end{tabular}\\
			\\
	\textbf{POST analysis}\\
	The Algorithm can be started via the REST-Interface using\\
			\begin{tabular}{lll}
				http & POST & \texttt{analytics/\{version\}/services/}\\
				& & \texttt{de.uni\_potsdam.hpi.bpt.bp2014.janalytics.\{classNameOfAlgorithm\}}
			\end{tabular}
			\\
			\\
If the Algorithm expects any parameters you have to put them in the request body as JSON Array with your POST. You have to keep the order in mind, the request body is the later input for the calculateResult(String[] args) method. \\
After the POST-Request a http 303 See Other with a new URI is returned. The URI (/jengine/api/analytics/v2/services/de.uni\_potsdam.hpi.bpt.bp2014.janalytics.\{class\\NameOfAlgorithm\}/resultId/\{resultId\}) can be used for a GET-Request to obtain the result of the calculation, the resultId is used to identify the result of the different POSTs.\\
\\
%%%----------
	\textbf{Get analysis results}\\
	After the Algorithm has terminated, the results can be obtained via 
	\\
			\begin{tabular}{lll}
				http & GET & \texttt{analytics/\{version\}/services/}\\
				& & \texttt{de.uni\_potsdam.hpi.bpt.bp2014.janalytics.\{classNameOfAlgorithm\}/}\\
				& & \texttt{resultId/\{resultId\}}
			\end{tabular}
			\\
			\\
			to the same URI as in the 303 See Other and returns a JSON with the output of the Algorithm.\\
It should be considered, that the Angular.js frontend will automatically send a GET Request to the URI of the http 303 See Other and no additional GET Request is needed when the output of an algorithm is integrated into the frontend.

		\begin{flushleft}
			\begin{lstlisting}
{
    "scenarioId":"163","meanScenarioInstanceRuntime":"0d:0h:3m:17s"
}

			\end{lstlisting}
			\captionof{json}{Example output of get request for  Example-Algorithm (meanScenarioInstanceRuntime)}
		\end{flushleft}

\newpage
	 
%%%%%%%%%%%%%%%%%%%%%%%%%% END %%%%%%%%%%%%%%%%%%%%%%%%%% 		
\end{document}
